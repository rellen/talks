\documentclass{beamer}
\beamertemplatenavigationsymbolsempty 
\usepackage{amsmath,amssymb,amsfonts}
\usepackage{amsmath}
%% \usepackage{tikz}
%% \usetikzlibrary{arrows,decorations.markings}

%% \usepackage{diagbox}
%% \usepackage{makecell}

%% \usepackage{graphicx}
\usepackage{svg}
\usepackage[procnames]{listings}

\usepackage{tex/latex-listings-protobuf/protobuf/lang}  % include language definition for protobuf
\usepackage{tex/latex-listings-protobuf/protobuf/style} % include custom style for proto declarations. 

\newcommand\lstproto[1]{%
  \lstinputlisting[language=protobuf2, style=protobuf, basicstyle=\tiny\ttfamily, numberstyle=\tiny]{#1}
}
\newcommand\lstprotonn[1]{%
  \lstinputlisting[language=protobuf2, style=protobuf, basicstyle=\tiny\ttfamily, numbers=none]{#1}
}

\lstdefinelanguage{elixir}{
  morekeywords={case,catch,def,do,else,false,%
    use,alias,receive,timeout,defmacro,defp,%
    for,if,import,defmodule,defprotocol,%
    nil,defmacrop,defoverridable,defimpl,%
    super,fn,raise,true,try,end,with,%
    unless},
  otherkeywords={<-,->, |>, \%\{, \}, \{, \, (, )},
  sensitive=true,
  morecomment=[l]{\#},
  morecomment=[n]{/*}{*/},
  morecomment=[s][\color{purple}]{:}{\ },
  morestring=[s][\color{orange}]"",
  commentstyle=\color[rgb]{0,1,0},
  keywordstyle=\color[rgb]{0,0,1},
  stringstyle=\color{red},
  procnamekeys={def,defp},
  procnamestyle=\color[rgb]{0,0.5,0},
  basicstyle=\ttfamily,
	breaklines,
	showstringspaces=false,
	frame=tb
}

\newcommand\lstelixir[1]{%
  \lstinputlisting[language=elixir, basicstyle=\large, numberstyle=\small]{#1}
}

%% \usepackage[default]{sourcecodepro}
\usefonttheme{professionalfonts}

\usepackage{sourcecodepro}
\usepackage[T1]{fontenc}
\usepackage[utf8]{inputenc}
%% Load the markdown package
\usepackage[citations,footnotes,definitionLists,hashEnumerators,smartEllipses,tightLists=false,pipeTables,tableCaptions,hybrid]{markdown}
%%begin novalidate
\markdownSetup{
  hybrid=true,
  underscores=false,
  rendererPrototypes={
   link = {\href{#2}{#1}},
   headingOne = {\section{#1}},
   headingTwo = {\subsection{#1}},
   headingThree = {\begin{frame}\frametitle{#1}},
   headingFour = {\begin{block}{#1}},
   horizontalRule = {\end{block}}
}}
%%end novalidate
\usefonttheme{structurebold}

\newcommand\blfootnote[1]{%
  \begingroup
  \renewcommand\thefootnote{}\footnote{#1}%
  \addtocounter{footnote}{-1}%
  \endgroup
}

\setbeamertemplate{footnote}{%
  \parindent 0em\noindent%
  \raggedleft%\raggedright
  \usebeamercolor{footnote}\hbox to 0.8em{\hfil\insertfootnotemark}\insertfootnotetext\par%
}

%Information to be included in the title page:
\title{gRPC in Elixir}
\author{Robert Ellen\\@robertellen}
\date{Elixir Australia - November 2020}
\titlegraphic{\includegraphics[width=0.2\textwidth]{../common/by.png}}

\begin{document}

\frame{\titlepage}

\begin{markdown}{hybrid}
%%begin novalidate

# Abstract
### Abstract

  - gRPC is `A high-performance, open source universal RPC framework'
  - inter-service communication
  - `last mile' communications to devices and browsers
  - any language: HTTP2, Protobuf
  - option in the landscape occupied by e.g. ReST, GraphQL, busses, SNMP
  - gRPC library for Elixir: Gun (client), Cowboy (server)  

\end{frame}


# Agenda
### Agenda

  - what is gRPC
  - who's using it
  - Elixr library (https://github.com/elixir-grpc/grpc)
 and example application

\end{frame}

# What is gRPC?
### What is gRPC

RPC framework

  - high performance
    * HTTP/2 transport
    * ProtoBuf for IDL and wire format 
  - multi-language 

\end{frame}

### Diagram 

\begin{center}
  \includesvg[width=250px]{grpc}
\end{center}

\blfootnote{https://grpc.io/img/landing-2.svg}

\end{frame}

### HTTP/2 features

  - binary data frames
  - compression
  - two-way streaming
  - TCP-connection multiplexing

\end{frame}

### Protocol Buffers

  - use as IDL provides a strong contract
  - codec generation
  - compact binary wire format

\end{frame}

### Protobuf messages

  \lstprotonn{example.proto}

  \blfootnote{https://grpc.io/docs/what-is-grpc/introduction/}

\end{frame}

### Protobuf IDL

  \lstproto{service.proto}

  \blfootnote{https://grpc.io/docs/what-is-grpc/introduction/}

\end{frame}

### RPC types 

  \lstproto{service_methods.proto}

  \blfootnote{https://grpc.io/docs/what-is-grpc/core-concepts/}

\end{frame}

### gRPC server

  - gRPC library generates a codec from `.proto` to native data structures
  - gRPC server must implement the services specified
  - listen on a particular port for a `channel`
  - service name and method name form a path
    * $\rightarrow$ load balancing, routing, etc

\end{frame}

### gRPC client

  - codec also generated for client
  - plus stubs of the service methods
  - client opens a channel to server
  - client calls stub like any other function/method

\end{frame}

# Who's using it?
### Who's using it?

  - inter-service communication
    * as an alternative to ReST, GraphQL, message busses, etc
  - APIs
    * Google
    * Cloud-native, k8s, etc
    * applications, e.g. Cockroach DB
  - Model-driven Telemetry (to supersede SNMP)
    * Cisco
    * Juniper

\end{frame}

### Example: Juniper Telemetry Interface

  \lstproto{telemetry_service.proto}

  \blfootnote{https://github.com/juniper/jtimon/blob/master/telemetry/telemetry.proto}

\end{frame}

### Example: Async messaging 

  \lstprotonn{async.proto}

  \blfootnote{https://pl-rants.net/posts/async-over-grpc/}

\end{frame}


# Elixir gRPC library
### Elixir gRPC library

  - https://github.com/elixir-grpc/grpc
  - client uses Gun
    * https://github.com/ninenines/gun
  - server uses Cowboy
    * https://github.com/ninenines/cowboy
  - separate protobuf library
    * https://github.com/tony612/protobuf-elixir

\end{frame}

### Demo time!

\end{frame}

### References

  - https://grpc.io/
  - https://github.com/elixir-grpc/grpc/
  - https://blog.appsignal.com/2020/03/24/how-to-use-grpc-in-elixir.html 
  - https://pl-rants.net/posts/async-over-grpc/
  - https://github.com/fullstorydev/grpcurl
    
\end{frame}


### Questions 

\end{frame}

### Licence

This work is licensed under the Creative Commons Attribution 4.0 International License. To view a copy of this license, visit http://creativecommons.org/licenses/by/4.0/ or send a letter to Creative Commons, PO Box 1866, Mountain View, CA 94042, USA.
 
\end{frame}

%%end novalidate
\end{markdown}


\end{document}
